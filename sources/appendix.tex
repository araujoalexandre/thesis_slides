%%%%%%%%%%%%%%%%%%%%%%%%%%%%%%%%%%%%%%%%%%%%%%%%%%%%%%%%%%%%%%%%%%%%%%%%%%%%%%%
\section{Appendix}
%%%%%%%%%%%%%%%%%%%%%%%%%%%%%%%%%%%%%%%%%%%%%%%%%%%%%%%%%%%%%%%%%%%%%%%%%%%%%%%

%%%%%%%%%%%%%%%%%%%%%%%%%%%%%%%%%%%%%%%%%%%%%%%%%%%%%%%%%%%%%%%%%%%%%%%%%%%%%%%
\begin{frame}{Efficient Matrix-vector product with Circulant Matrices}
%%%%%%%%%%%%%%%%%%%%%%%%%%%%%%%%%%%%%%%%%%%%%%%%%%%%%%%%%%%%%%%%%%%%%%%%%%%%%%%

  A circulant matrix $\Cmat \in \Rbb^{n \times n}$ such as $\Cmat = \circulant (\cvec)$, with $\cvec \in \Rbb^n$ can be diagonalized by the Discrete Fourier Transform:
  \begin{equation*}
      \Cmat = \Wmat^{-1} \Lambda \Wmat
  \end{equation*}
  where $\Wmat = \frac{1}{\sqrt{n}} \left( \omega^{jk} \right)_{j,k = 0, \dots, n-1}$ with $\omega$ being the $n^{th}$ root of unity, $\Lambda$ is a diagonal matrix with the eigenvalues of the matrix $\Cmat$ and the eigenvalues of the matrix $\Cmat$ can correspond to $\Wmat \cvec$. 

  Therefore, thanks to the convolution theorem, matrix-vector multiplication can be done efficiently with the \textbf{Fast Fourier Transform} as follows:
  \begin{equation*}
    \Cmat \xvec = \mathrm{IDFT}( \mathrm{DFT}(\cvec) * \mathrm{DFT}(\xvec) )
  \end{equation*}
  where the multiplication is performed elements-wise. 
\end{frame}
