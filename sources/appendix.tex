%%%%%%%%%%%%%%%%%%%%%%%%%%%%%%%%%%%%%%%%%%%%%%%%%%%%%%%%%%%%%%%%%%%%%%%%%%%%%%%
\section{Appendix}
%%%%%%%%%%%%%%%%%%%%%%%%%%%%%%%%%%%%%%%%%%%%%%%%%%%%%%%%%%%%%%%%%%%%%%%%%%%%%%%



%%%%%%%%%%%%%%%%%%%%%%%%%%%%%%%%%%%%%%%%%%%%%%%%%%%%%%%%%%%%%%%%%%%%%%%%%%%%%%%
\begin{frame}{Efficient Matrix-vector product with Circulant Matrices}
%%%%%%%%%%%%%%%%%%%%%%%%%%%%%%%%%%%%%%%%%%%%%%%%%%%%%%%%%%%%%%%%%%%%%%%%%%%%%%%

  A circulant matrix $\Cmat \in \Rbb^{n \times n}$ such as $\Cmat = \circulant (\cvec)$, with $\cvec \in \Rbb^n$ can be diagonalized by the Discrete Fourier Transform:
  \begin{equation*}
      \Cmat = \Wmat^{-1} \Lambda \Wmat
  \end{equation*}
  where $\Wmat = \frac{1}{\sqrt{n}} \left( \omega^{jk} \right)_{j,k = 0, \dots, n-1}$ with $\omega$ being the $n^{th}$ root of unity, $\Lambda$ is a diagonal matrix with the eigenvalues of the matrix $\Cmat$ and the eigenvalues of the matrix $\Cmat$ can correspond to $\Wmat \cvec$. 

  Therefore, thanks to the convolution theorem, matrix-vector multiplication can be done efficiently with the \textbf{Fast Fourier Transform} as follows:
  \begin{equation*}
    \Cmat \xvec = \mathrm{IDFT}( \mathrm{DFT}(\cvec) * \mathrm{DFT}(\xvec) )
  \end{equation*}
  where the multiplication is performed elements-wise. 
\end{frame}




%%%%%%%%%%%%%%%%%%%%%%%%%%%%%%%%%%%%%%%%%%%%%%%%%%%%%%%%%%%%%%%%%%%%%%%%%%%%%%%%
\begin{frame}{Adversarial attacks}
%%%%%%%%%%%%%%%%%%%%%%%%%%%%%%%%%%%%%%%%%%%%%%%%%%%%%%%%%%%%%%%%%%%%%%%%%%%%%%%%

  \begin{minipage}{\textwidth}
   \centering
   An \textbf{adversarial attack} refers to a small change of an input maliciously designed to fool the result of a neural network. 
  \end{minipage}
  \vspace{0.7cm}

  \only<1>{
    \begin{minipage}{\textwidth}
      \centering
      \begin{overpic}[trim=0 1.2cm 0 0.3cm,clip,width=0.9\textwidth]{images/ExampleAdversarialCatDog.pdf}
	\put (7.5, 19) {
	  \footnotesize Image
	}
	\put (3.5, -3) {
	  \footnotesize Label = ``cat''
	}
	\put (25, -2) {
	  \begin{tikzpicture}
	    \draw[color=white,fill] (0,0) -- (8,0) -- (8,2.5) -- (0,2.5) -- (0,0);
	  \end{tikzpicture}
	 }
      \end{overpic}
    \end{minipage}
  }

  \only<2>{
    \begin{minipage}{\textwidth}
      \centering
      \begin{overpic}[trim=0 1.2cm 0 0.3cm,clip,width=0.9\textwidth]{images/ExampleAdversarialCatDog.pdf}
	\put (7.5, 19) {
	  \footnotesize Image
	}
	\put (39.5, 19) {
	  \footnotesize Adversarial Perturbation
	}
	\put (3.5, -3) {
	  \footnotesize Label = ``cat''
	}
      \put (67, -2) {
	\begin{tikzpicture}
	  \draw[color=white,fill] (0,0) -- (3.1,0) -- (3.1,1.9) -- (0,1.9) -- (0,0);
	\end{tikzpicture}
       }
      \end{overpic}
    \end{minipage}
  }

  \only<3-5>{
    \begin{minipage}{\textwidth}
      \centering
      \begin{overpic}[trim=0 1.2cm 0 0.3cm, clip, width=0.9\textwidth]{images/ExampleAdversarialCatDog.pdf}
	\put (7.5, 19) {
	  \footnotesize Image
	}
	\put (39.5, 19) {
	  \footnotesize Adversarial Perturbation
	}
	\put (78, 19) {
	  \footnotesize Adversarial Image
	}
	\put (3.5, -3) {
	  \footnotesize Label = ``cat''
	}
	\put (80, -3) {
	  \footnotesize Label = ``dog''
	}
      \end{overpic}
    \end{minipage}
  }

  \vspace{0.15cm}
  \visible<4-5>{
    \begin{itemize}
      \item[$\bullet$] \orangebold{Numerous} methods exist to craft an \orangebold{adversarial perturbation};
      \item[$\bullet$] The best attacks reduce the accuracy of an undefended model to \orangebold{0\%};
    \end{itemize}
  }

  \vspace{0.5cm}
  \visible<5>{
    \begin{mdframed}[linecolor=OrangePSL,linewidth=1pt]
      \centering
      Adversarial examples poses a growing societal problem as more and more machine learning models are deployed into critical-decision systems.
    \end{mdframed}
  }

\end{frame}
