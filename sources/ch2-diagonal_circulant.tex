%%%%%%%%%%%%%%%%%%%%%%%%%%%%%%%%%%%%%%%%%%%%%%%%%%%%%%%%%%%%%%%%%%%%%%%%%%%%%%%
\section{Diagonal Circulant Neural Networks}
%%%%%%%%%%%%%%%%%%%%%%%%%%%%%%%%%%%%%%%%%%%%%%%%%%%%%%%%%%%%%%%%%%%%%%%%%%%%%%%

%%%%%%%%%%%%%%%%%%%%%%%%%%%%%%%%%%%%%%%%%%%%%%%%%%%%%%%%%%%%%%%%%%%%%%%%%%%%%%%
\subsection{Expressivity of Diagonal Circulant Neural Networks}
%%%%%%%%%%%%%%%%%%%%%%%%%%%%%%%%%%%%%%%%%%%%%%%%%%%%%%%%%%%%%%%%%%%%%%%%%%%%%%%


% %%%%%%%%%%%%%%%%%%%%%%%%%%%%%%%%%%%%%%%%%%%%%%%%%%%%%%%%%%%%%%%%%%%%%%%%%%%%%%%
% \begin{frame}{Model compression for Deep Learning}
% %%%%%%%%%%%%%%%%%%%%%%%%%%%%%%%%%%%%%%%%%%%%%%%%%%%%%%%%%%%%%%%%%%%%%%%%%%%%%%%
%
%   \textbf{Constraining the weight representation}
%   \begin{itemize}
%     \item Floating variable with limited precision \cite{Gupta:2015:DLL:3045118.3045303}
%     \item Quantization \cite{courbariaux2015binaryconnect,MellempudiKM0KD17, rastegariECCV16}
%     \item Hashing techniques \cite{chen2015hashing}
%   \end{itemize}
%    
%   \textbf{Matrix factorization}
%   \begin{itemize}
%     \item Use of low rank matrices and decomposition as weights matrices \cite{NIPS2013_5025, Jaderberg2014SpeedingUC, 8099498}
%   \end{itemize}
%
%   \textbf{Imposing structures on weight matrices}
%   \begin{itemize}
%       \item Use structured matrices instead of Dense Matrices \cite{7410684,NIPS2015_5869}
%   \end{itemize}
%
% \end{frame}


%%%%%%%%%%%%%%%%%%%%%%%%%%%%%%%%%%%%%%%%%%%%%%%%%%%%%%%%%%%%%%%%%%%%%%%%%%%%%%%
\begin{frame}{Circulant matrices for Deep Learning}
%%%%%%%%%%%%%%%%%%%%%%%%%%%%%%%%%%%%%%%%%%%%%%%%%%%%%%%%%%%%%%%%%%%%%%%%%%%%%%%

  A $n \times n$ circulant matrix $\Cmat$ is a matrix  where each row is a cyclic right shift of the previous one as illustrated below.
  \begin{equation*}
      \Cmat = \circulant (\cvec) = 
      \begin{pmatrix} \vspace{0.1cm}
      c_0 & c_1 & c_2 & \dots & c_{n-1} \\ \vspace{0.1cm}
      c_{n-1} & c_0 & c_2  & & c_{n-2} \\ \vspace{0.1cm}
      c_{n-2} & c_{n-1} & c_{0} & & c_{n-3} \\ \vspace{0.1cm}
      \vdots & & & \ddots  & \vdots \\ \vspace{0.1cm}
      c_1 & c_2 & c_3 & & c_0
      \end{pmatrix}
  \end{equation*}
  \begin{block}{Advantages:}
    \begin{itemize}
      \small
      \item[$\bullet$] A $n \times n$ circulant matrix can be \textbf{compactly represented in memory} using only $n$ real values.
      \item[$\bullet$] Multiplying a circulant matrix $\Cmat$ by a vector $\xvec$ \textbf{can be done efficiently in the Fourier domain}
    \end{itemize}
  \end{block}
  \begin{block}{Limits:}
    \begin{itemize}
      \small
      \item[$\bullet$] The product of circulant matrices is not expressive: circulant matrices are closed under product
    \end{itemize}
  \end{block}

\end{frame}



% %%%%%%%%%%%%%%%%%%%%%%%%%%%%%%%%%%%%%%%%%%%%%%%%%%%%%%%%%%%%%%%%%%%%%%%%%%%%%%%
% \begin{frame}{Efficient Matrix-vector product with Circulant Matrices}
% %%%%%%%%%%%%%%%%%%%%%%%%%%%%%%%%%%%%%%%%%%%%%%%%%%%%%%%%%%%%%%%%%%%%%%%%%%%%%%%
%     
%   A circulant matrix $\Cmat \in \Rbb^{n \times n}$ such as $\Cmat = \circulant (\cvec)$, with $\cvec \in \Rbb^n$ can be diagonalized by the Discrete Fourier Transform:
%   \begin{equation*}
%       \Cmat = \Wmat^{-1} \Lambda \Wmat
%   \end{equation*}
%   where $\Wmat = \frac{1}{\sqrt{n}} \left( \omega^{jk} \right)_{j,k = 0, \dots, n-1}$ with $\omega$ being the $n^{th}$ root of unity, $\Lambda$ is a diagonal matrix with the eigenvalues of the matrix $\Cmat$ and the eigenvalues of the matrix $\Cmat$ can correspond to $\Wmat \cvec$. 
%   
%   Therefore, thanks to the convolution theorem, matrix-vector multiplication can be done efficiently with the \textbf{Fast Fourier Transform} as follows:
%   \begin{equation*}
%     \Cmat \xvec = \mathrm{IDFT}( \mathrm{DFT}(\cvec) * \mathrm{DFT}(\xvec) )
%   \end{equation*}
%   where the multiplication is performed elements-wise. 
% \end{frame}


% \begin{frame}{Circulant matrices for Deep Learning}

% The fully connected layers are then represented as follows:
% \begin{equation*}
% 	h(x) = \phi\left(\left[\prod_{i=1}^{k} D^{(i)} C^{(i)}\right]x + b\right)
% \end{equation*}
% \noindent
% where the parameters of each matrix $D^{(i)}$ and $C^{(i)}$ are trained using a gradient based optimization algorithm, and $k$ defines the number of factors we choose for the training. 
% \end{frame}



%%%%%%%%%%%%%%%%%%%%%%%%%%%%%%%%%%%%%%%%%%%%%%%%%%%%%%%%%%%%%%%%%%%%%%%%%%%%%%%
\begin{frame}{Expressivity of the product of diagonal and circulant matrices}
%%%%%%%%%%%%%%%%%%%%%%%%%%%%%%%%%%%%%%%%%%%%%%%%%%%%%%%%%%%%%%%%%%%%%%%%%%%%%%%

 
  \begin{theorem}[Reformulation from \citet{Huhtanen2015}] 
    For every matrix $\Amat \in \Cbb^{n \times n}$, for any $\epsilon > 0$, there exists a sequence of matrices $\Bmat_1 \cdots \Bmat_{2n-1}$ where $\Bmat_i$ is a circulant matrix if $i$ is odd, and a diagonal matrix otherwise, such that $\norm{\Bmat_1 \Bmat_2 \cdots \Bmat_{2n-1} - \Amat} < \epsilon$.
  \end{theorem}

  \vspace{0.5cm}

  \begin{itemize}
    \item[$\bullet$] The decomposition needs more values that $n^2$
    \item[$\bullet$] The theorem does not provide any insights regarding the expressive power of $m$ diagonal-circulant factors when $m$ is much lower than $2n - 1$
  \end{itemize}


  % This theorem is of little use to understand the expressive power of diagonal-circulant matrices when they are used in deep neural networks:
  % \begin{itemize}
  %     \item The bound only depends on the dimension of the matrix $\Amat$
  %     \item The theorem does not provide any insights regarding the expressive power of $m$ diagonal-circulant factors when $m$ is much lower than $2n - 1$
  % \end{itemize}

\end{frame}


%%%%%%%%%%%%%%%%%%%%%%%%%%%%%%%%%%%%%%%%%%%%%%%%%%%%%%%%%%%%%%%%%%%%%%%%%%%%%%%
\begin{frame}{Relation between diagonal circulant matrices and low rank matrices}
%%%%%%%%%%%%%%%%%%%%%%%%%%%%%%%%%%%%%%%%%%%%%%%%%%%%%%%%%%%%%%%%%%%%%%%%%%%%%%%
      
  \begin{theorem}[Rank-based circulant decomposition] Let $\Amat \in \mathbb{C}^{n \times n}$ be
  a matrix of rank at most $k$. Assume that $n$ can be divided by $k$. For
  any $\epsilon > 0$, there exists a sequence of $4k+1$ matrices $\Bmat_{1}, \dots, \Bmat_{4k+1},$ where $\Bmat_{i}$ is a circulant matrix if $i$ is odd, and a diagonal matrix otherwise, such that $\norm{\Bmat_1\Bmat_2 \dots \Bmat_{4k+1} - \Amat} < \epsilon$
  \end{theorem}

  \vspace{0.5cm}

  % A direct consequence of this theorem, is that if the number of diagonal-circulant factors is set to a value $K$, we can represent all linear transform whose rank is $\frac{K - 1}{4}$.

  \begin{itemize}
    \item[$\Rightarrow$] If the number of diagonal-circulant factors is set to a value $K$, we can represent all linear transform whose rank is $\frac{K - 1}{4}$.
  \end{itemize}
\end{frame}


%%%%%%%%%%%%%%%%%%%%%%%%%%%%%%%%%%%%%%%%%%%%%%%%%%%%%%%%%%%%%%%%%%%%%%%%%%%%%%%
\begin{frame}{Diagonal-Circulant Neural Network}
%%%%%%%%%%%%%%%%%%%%%%%%%%%%%%%%%%%%%%%%%%%%%%%%%%%%%%%%%%%%%%%%%%%%%%%%%%%%%%%

  We replace the weight matrices of Fully-Connected layers by a product of Diagonal and Circulant matrices :

  \begin{block}{Fully-Connected layer}
  \begin{equation*}
      \xvec \mapsto \Wmat \xvec + \bvec
  \end{equation*}
  where $\xvec \in \Rbb^n$, $\bvec \in \Rbb^n$, $\Wmat \in \Rbb^{n \times n}$.
  \end{block}

  \begin{block}{Diagonal-Circulant layer}
  \begin{equation*}
      \xvec \mapsto \left[ \prod_{i=0}^{k} \Dmat_i \Cmat_i \right] \xvec + \bvec
  \end{equation*}
  where $\xvec \in \Rbb^n$, $\bvec \in \Rbb^n$, $\Dmat_i \in \Rbb^{n \times n}$ is a diagonal matrix, $\Cmat_i \in \Rbb^{n \times n}$ is a circulant matrix, $k$ is a user defined parameter.
  \end{block}

\end{frame}





% \begin{frame}{Diagonal-Circulant Neural Network}
% Instead of choosing a parameter $k$ for each layer, we study the Diagonal-Circulant architecture:
% \begin{block}{Diagonal-Circulant neural network}
%     A diagonal-Circulant neural network of $\ell$ layers is defined as follows:
%     \begin{equation*}
%         \mathcal{N}(x) = \phi^{(\ell)}_{\W}_\ell,\bvec_\ell} \circ \phi^{(\ell-1)}_{\Wmat_\ell-1,\bvec_\ell-1} \circ \cdots \circ \phi^{(1)}_{\Wmat_1,\bvec}(x)
%     \end{equation*}
%      where for any $i$, $\phi^{(i)}_{\D}_i, \Cmat_i,\bvec_i} := z \mapsto \rho(\Dmat_i \Cmat_i \mathbf{z} + \bvec_i)$, $\mathbf{z}_i \in \Rbb^m$, $\bvec_i \in \Rbb^m$, $\Dmat_i \in \Rbb^{m \times n}$, $\Cmat_i \in \Rbb^{m \times n}$, and $\rho$ some non linear (activation) function.
% \end{block}
% \end{frame}




%%%%%%%%%%%%%%%%%%%%%%%%%%%%%%%%%%%%%%%%%%%%%%%%%%%%%%%%%%%%%%%%%%%%%%%%%%%%%%%
\begin{frame}{Expressive Power of Diagonal-Circulant Neural Network}
%%%%%%%%%%%%%%%%%%%%%%%%%%%%%%%%%%%%%%%%%%%%%%%%%%%%%%%%%%%%%%%%%%%%%%%%%%%%%%%

  \begin{theorem}[Rank-based expressive power of DCNNs] \label{th:rank-based_dcnn}
    Let $\mathcal{N}$ be a deep ReLU network of width $n$, depth $L$ and a total rank $k$\footnote{The sum of ranks of the weight matrices}.
    Let $\mathcal{X} \subset \Cbb^{n}$ be a bounded set.
    Then, for any $\epsilon>0$, there exists a DCNN with ReLU activation $\mathcal{N}'$ of width $n$ such that $\norm{\mathcal{N}(\xvec) - \mathcal{N}'(\xvec) } < \epsilon$ for all $\xvec \in \mathcal{X}$ and the depth of $\mathcal{N}'$ is bounded by $9k$.
  \end{theorem}

  By combining Theorem~\ref{th:rank-based_dcnn} and the universal approximation theorem of Neural Network, we have:
  \begin{corollary} \label{th:universal}
    Bounded width DCNNs are \textbf{universal approximators} 
  \end{corollary}

\end{frame}



%%%%%%%%%%%%%%%%%%%%%%%%%%%%%%%%%%%%%%%%%%%%%%%%%%%%%%%%%%%%%%%%%%%%%%%%%%%%%%%
\subsection{Experiments: Large Scale Video Classification}
%%%%%%%%%%%%%%%%%%%%%%%%%%%%%%%%%%%%%%%%%%%%%%%%%%%%%%%%%%%%%%%%%%%%%%%%%%%%%%%



%%%%%%%%%%%%%%%%%%%%%%%%%%%%%%%%%%%%%%%%%%%%%%%%%%%%%%%%%%%%%%%%%%%%%%%%%%%%%%%
\begin{frame}{Experimental Setup}
%%%%%%%%%%%%%%%%%%%%%%%%%%%%%%%%%%%%%%%%%%%%%%%%%%%%%%%%%%%%%%%%%%%%%%%%%%%%%%%

  \begin{block}{Dataset: \yt}
  \begin{itemize}
    \item 8 millions embedded audio \& video frames
    \item 3200 classes
  \end{itemize}
  \end{block}
  % The network randomly samples video and audio frames from the input. The sample goes through an embedding layer and is reduced with a Fully Connected layer. The results are then concatenated and classified with a Mixture-of-Experts and a Context Gating layer.
  State-of-the-art architecture for video classification (\cite{miech2017youtube}).
  \begin{figure}[htb]
    \scalebox{0.65}{\tikzset{%
  >={Latex[width=2mm,length=2mm]},
  % Specifications for style of nodes:
            base/.style = {rectangle, draw=black, text centered, font=\sffamily},
             box/.style = {base, rounded corners, text depth=3cm, minimum height=4cm, minimum width=3cm},
     transparent/.style = {rectangle, draw=black},
       circulant/.style = {base, fill=yellow!30},
       embedding/.style = {base, fill=blue!30, minimum width=2.5cm, minimum height=1cm},
           other/.style = {base, fill=white!30,  minimum width=2cm, minimum height=1cm},
              fc/.style = {base, fill=orange!30, minimum width=1.5cm, minimum height=1cm},
          gating/.style = {base, fill=green!30, minimum width=2cm, text width=2cm, minimum height=1cm},
             moe/.style = {base, fill=purple!30, minimum width=1.5cm, minimum height=1cm},
}

\begin{tikzpicture}[every node/.style={fill=white, font=\sffamily}, align=center]

  % \draw (0.0, +2.)  node [other, draw=none, opacity=0, text opacity=1] {\textbf{Embedding}};
  % \draw (+3.7, +2.)  node [other, draw=none, opacity=0, text opacity=1] {\textbf{Dim Reduction}};
  % \draw (+8.0, +2.)  node [other, draw=none, opacity=0, text opacity=1] {\textbf{Classification}};
  \draw (0.0, +2.)  node [other, draw=none, opacity=0, text opacity=1] {\textbf{Layer 1}};
  \draw (+3.7, +2.)  node [other, draw=none, opacity=0, text opacity=1] {\textbf{Layer 2}};
  \draw (+8.0, +2.)  node [other, draw=none, opacity=0, text opacity=1] {\textbf{Layer 3}};

  \draw (0, +0.8)  node [embedding] {Video};
  \draw (0, -0.8)  node [embedding] {Audio};

  \draw (+2.5, +0.8)  node (fc) [fc] {FC};
  \draw (+2.5, -0.8)  node (fc) [fc] {FC};

  \draw (+4.75, 0)  node (fc) [other] {concat};
  \draw (+7.0, 0)  node (moe) [moe] {MoE};
  \draw (+9.25, 0)  node (gating2) [gating] {Context Gating};
 
  \draw (+1.5, +2) [dashed] -- (+1.5, -1.7);
  \draw (+6, +2) [dashed] -- (+6, -1.7);
  
  % \draw (3.5, -2.6)  node [other, draw=none, opacity=0, text opacity=1] {\textbf{use of Diagonal-Circulant layers}};
  % \draw (0.0, -1.5) -- (2.7, -2.3);
  % \draw (2.5, -1.5) -- (3.3, -2.3);
  % \draw (7.0, -0.8) -- (4.0, -2.3);
  
\end{tikzpicture}
}
  \end{figure}
  $\Rightarrow$ This architecture has $5.7$ millions parameters.

\end{frame}


%%%%%%%%%%%%%%%%%%%%%%%%%%%%%%%%%%%%%%%%%%%%%%%%%%%%%%%%%%%%%%%%%%%%%%%%%%%%%%%
\begin{frame}{Effect of Diagonal-Circulant layers}
%%%%%%%%%%%%%%%%%%%%%%%%%%%%%%%%%%%%%%%%%%%%%%%%%%%%%%%%%%%%%%%%%%%%%%%%%%%%%%%

  \begin{figure}[!htb]
    \centering
    \scalebox{0.8}{\begin{tikzpicture}
\begin{axis}[
    width=0.85\textwidth,
    height=0.6\textwidth,
    title={\parbox{7cm}{\centering \textbf{Comparison of the effect of compactness over different layers with the base model}}},
    legend style={draw=black},
    legend cell align={left},
    xlabel={Epochs},
    ylabel={Validation GAP},
    xmin=0, xmax=7,
    ymin=0.77, ymax=0.86,
    xtick={0,1,2,3,4,5,6,7},
    ytick={0.77,0.78,0.79,0.8,0.81,0.82,0.83,0.84,0.85,0.86},
    legend pos=outer north east,
    ymajorgrids=true,
    grid style=dashed,
	]
  \addplot[color=red] table [y=gap, x=epoch]{data/layers/dense.dat};
  \addplot[color=yellow] table [y=gap, x=epoch]{data/layers/compact_dbof.dat};
  \addplot[color=blue] table [y=gap, x=epoch]{data/layers/compact_fc.dat};
  \addplot[color=green] table [y=gap, x=epoch]{data/layers/compact_moe.dat};
  \draw [<-] (axis cs:2.0,0.834) -- +(-10pt,+10pt) node[left] {9.2\%};
  \draw [<-] (axis cs:2.8,0.835) -- +(+10pt,-10pt) node[right] {original};
  \draw [<-] (axis cs:5.2,0.830) -- +(+10pt,-10pt) node[right] {18.4\%};
  \draw [<-] (axis cs:5.0,0.800) -- +(+10pt,-10pt) node[right] {72.0\%};

\legend{
   Dense Model,
   Model w/compact DBoF, 
   Model w/compact FC, 
   Model w/compact MoE,
 }
\end{axis}
\end{tikzpicture}}
  \end{figure}
  % Validation GAP according to the number of epochs for different compact models.
  $\Rightarrow$ 9.2\% compression rate without loss in accuracy \\	
  $\Rightarrow$ 72\% compression rate with a loss of 4 points in accuracy
\end{frame}

%%%%%%%%%%%%%%%%%%%%%%%%%%%%%%%%%%%%%%%%%%%%%%%%%%%%%%%%%%%%%%%%%%%%%%%%%%%%%%%
% \begin{frame}{Effect of circulant matrices with different embeddings}
% %%%%%%%%%%%%%%%%%%%%%%%%%%%%%%%%%%%%%%%%%%%%%%%%%%%%%%%%%%%%%%%%%%%%%%%%%%%%%%%
%
%   The figures below show the validation GAP of \textmd{compact} and \textit{Dense} fully connected layer with different embeddings according to the number of epochs.
%   \large
%   \begin{figure}[!htb]
%     \centering
%     \captionsetup{justification=justified}
%     \scalebox{0.7}{\begin{tikzpicture}[scale=0.58]
\begin{axis}[
    title={\large \textbf{DBoF}},
    legend style={fill=none,draw=none},
    legend cell align={left},
    xlabel={Epochs},
    ylabel={Validation GAP},
    xmin=0, xmax=10,
    ymin=0.81, ymax=0.88,
    xtick={0,1,2,3,4,5,6,7,8,9,10},
    ytick={0.81, 0.82, 0.83,0.84,0.85,0.86, 0.87, 0.88},
    legend pos=north west,
    ymajorgrids=true,
    grid style=dashed,
	]
  \addplot[color=blue] table [y=gap, x=epoch]{data/fc_embedding/dbof_compressed.dat};
  \addplot[color=red] table [y=gap, x=epoch]{data/fc_embedding/dbof_uncompressed.dat};
  \draw [<-] (axis cs:2.0,0.85) -- +(+20pt,-20pt) node[right] {9.2\%};
\legend{Compact, Dense}
\end{axis}
\end{tikzpicture}
\begin{tikzpicture}[scale=0.58]
\begin{axis}[
    title={\large \textbf{NetVLAD}},
    legend style={fill=none,draw=none},
    legend cell align={left},
    xlabel={Epochs},
    ylabel={Validation GAP},
    xmin=0, xmax=10,
    ymin=0.80, ymax=0.88,
    xtick={0,1,2,3,4,5,6,7,8,9,10},
    ytick={0.81, 0.82, 0.83,0.84,0.85,0.86, 0.87, 0.88},
    legend pos=north west,
    ymajorgrids=true,
    grid style=dashed,
  ]
  \addplot[color=blue] table [y=gap, x=epoch]{data/fc_embedding/netvlad_compressed.dat};
  \addplot[color=red] table [y=gap, x=epoch]{data/fc_embedding/netvlad_uncompressed.dat};
  \draw [<-] (axis cs:3.0,0.835) -- +(+10pt,-10pt) node[right] {41.1\%};
\legend{Compact, Dense}
\end{axis}
\end{tikzpicture}
\begin{tikzpicture}[scale=0.58]
\begin{axis}[
    title={\large \textbf{NetFV}},
    legend style={fill=none,draw=none},
    legend cell align={left},
    xlabel={Epochs},
    ylabel={Validation GAP},
    xmin=0, xmax=10,
    ymin=0.810, ymax=0.88,
    xtick={0,1,2,3,4,5,6,7,8,9,10},
    ytick={0.81, 0.82, 0.83, 0.84, 0.85,0.86, 0.87, 0.88},
    legend pos=north west,
    ymajorgrids=true,
    grid style=dashed,
  ]
  \addplot[color=blue] table [y=gap, x=epoch]{data/fc_embedding/fisher_compressed.dat};
  \addplot[color=red] table [y=gap, x=epoch]{data/fc_embedding/fisher_uncompressed.dat};
  \draw [<-] (axis cs:5.0,0.841) -- +(+10pt,-10pt) node[right] {58.1\%};
\legend{Compact, Dense}
\end{axis}
\end{tikzpicture}}
%   \end{figure}
% \end{frame}



% %%%%%%%%%%%%%%%%%%%%%%%%%%%%%%%%%%%%%%%%%%%%%%%%%%%%%%%%%%%%%%%%%%%%%%%%%%%%%%%
% \begin{frame}{Results}
% %%%%%%%%%%%%%%%%%%%%%%%%%%%%%%%%%%%%%%%%%%%%%%%%%%%%%%%%%%%%%%%%%%%%%%%%%%%%%%%
%
%   % \begin{table}
%   %   \centering
%   %   \caption{\small
%   %      This table shows the GAP score for the \yt dataset with DCNNs.
%   %      We can see a large increase in the score with deeper networks.}
%   %   \small
%   %   \begin{tabular}{lccc}
%   %     \toprule
%   %     \textbf{Architecture} & \textbf{\#Weights} &
%   %     \textbf{GAP@20} \\
%   %     \hline \\
%   %     \textit{original} & \textit{5.7M} & \textit{0.773} \\
%   %     4 DC & 25 410  (\textit{\bf 0.44}) & 0.599   \\
%   %     32 DC  & 122 178 \textit{(2.11)} & 0.685   \\
%   %     4 DC + 1 FC & 4.46M \textit{(77)} & \textbf{0.747} \\
%   %   \hline
%   %   \end{tabular}
%   %   \label{table:youtube_agg_xp}
%   % \end{table}
%
%
%   \begin{table}
%     \centering
%     \caption{\small
%       This table shows the GAP score for the \yt dataset with different layers represented with our DC decomposition.}
%     \small
%     \begin{tabular}{lccc}
%     \toprule
%     \textbf{Architecture} & \textbf{\#Weights} & \textbf{GAP@20} \\
%     \hline \\
%     \textit{original} & \textit{45M} & \textit{0.846} \\
%     DBoF with DC   & 36M (\textit{80}) & 0.838 \\
%     FC with DC    & 41M (\textit{91}) & \textbf{0.845} \\
%     MoE with DC   & 12M (\textit{\bf 26}) & 0.805 \\
%     \hline
%     \end{tabular}
%     \label{table:youtube_full_xp}
%   \end{table}
%
% \end{frame}



