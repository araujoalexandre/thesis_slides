%%%%%%%%%%%%%%%%%%%%%%%%%%%%%%%%%%%%%%%%%%%%%%%%%%%%%%%%%%%%%%%%%%%%%%%%%%%%%%%
\section{\orangebold{Part 1.} Compact Neural Networks with Diagonal and Circulant Matrices}
%%%%%%%%%%%%%%%%%%%%%%%%%%%%%%%%%%%%%%%%%%%%%%%%%%%%%%%%%%%%%%%%%%%%%%%%%%%%%%%



%%%%%%%%%%%%%%%%%%%%%%%%%%%%%%%%%%%%%%%%%%%%%%%%%%%%%%%%%%%%%%%%%%%%%%%%%%%%%%%
\begin{frame}{Circulant matrices for Deep Learning}
%%%%%%%%%%%%%%%%%%%%%%%%%%%%%%%%%%%%%%%%%%%%%%%%%%%%%%%%%%%%%%%%%%%%%%%%%%%%%%%

  Recall the Fully-Connected layer:
  \begin{equation}
    \xvec \mapsto \rho\left( {\color<2->{OrangePSL}{\Wmat}} \xvec + \bvec \right)
  \end{equation}

  \vspace{-0.4cm}
  where $\Wmat \in \Rbb^{n \times n}$, $\bvec \in \Rbb^n$.

  \vspace{0.3cm}
  \visible<2->{
    \begin{mdframed}[linecolor=OrangePSL,linewidth=1pt]
      \centering
      % \textbf{Goal}: We want to reduce the number of parameters of the layer. \\
       Can we replace the dense matrix $\Wmat$ with a structured one ? 
    \end{mdframed}
  } 
   
  \vspace{0.3cm}
  \visible<3->{
    Circulant matrices have numerous advantages:
    \begin{itemize}
	\item[$\bullet$] <3-> A circulant matrix can be \orangebold{compactly represented in memory};
	\item[$\bullet$] <3-> The matrix-vector product with a circulant matrix \orangebold{can be done efficiently in the Fourier domain}
        \item[$\Rightarrow$] <4-> They are not expressive: circulant matrices are closed under product.
    \end{itemize}
  }


\end{frame}



%%%%%%%%%%%%%%%%%%%%%%%%%%%%%%%%%%%%%%%%%%%%%%%%%%%%%%%%%%%%%%%%%%%%%%%%%%%%%%%
\begin{frame}{Expressivity of the product of diagonal and circulant matrices}
%%%%%%%%%%%%%%%%%%%%%%%%%%%%%%%%%%%%%%%%%%%%%%%%%%%%%%%%%%%%%%%%%%%%%%%%%%%%%%%

  \begin{minipage}{\textwidth}
    \centering
    Combining circulant matrices with \orangebold{diagonal matrices} improve the expressivity.
  \end{minipage}
  \vspace{0.05cm}

  \visible<2->{
    \begin{theorem}[Reformulation from {\color{SkyBlue}{\citet{huhtanen2015factoring}}}] 
      For every matrix $\Amat \in \Cbb^{n \times n}$, for any $\epsilon > 0$, there exists a sequence of circulant matrices and a sequence of diagonal matrices such that 
      \begin{equation}
	\norm{\prod_{i=1}^{n+1} \Dmat^{(i)} \Cmat^{(i)} - \Amat}_{\mathrm{F}} < \epsilon \enspace.
      \end{equation}
    \end{theorem}
  }

  {\small
  \visible<3->{\textbf{Advantages}}
  \begin{itemize}[parsep=0pt,leftmargin=15pt]
    \nointerlineskip
    \item[$\bullet$] <3-> Neural networks with Diagonal and Circulant Layer are \textbf{universal approximators}; 
  \end{itemize}

  \visible<4->{\textbf{Limits}}
  \begin{itemize}[parsep=0pt,leftmargin=15pt]
    \nointerlineskip
    \item[$\bullet$] <4-> The decomposition needs more values that $n^2$
    \item[$\bullet$] <5-> The theorem does not provide any insights regarding the expressive power of $m$ diagonal-circulant factors when $m$ is much lower than $n + 1$
  \end{itemize}
  }

  % By combining this result and the universal approximation theorem of Neural Network ({\color{SkyBlue}{\cite{hanin2017universal}}}), we have the following result:
  % \begin{theorem}
  %   Bounded width DCNNs are \textbf{universal approximators} 
  % \end{theorem}


  % This theorem is of little use to understand the expressive power of diagonal-circulant matrices when they are used in deep neural networks:
  % \begin{itemize}
  %     \item The bound only depends on the dimension of the matrix $\Amat$
  %     \item The theorem does not provide any insights regarding the expressive power of $m$ diagonal-circulant factors when $m$ is much lower than $2n - 1$
  % \end{itemize}

\end{frame}


%%%%%%%%%%%%%%%%%%%%%%%%%%%%%%%%%%%%%%%%%%%%%%%%%%%%%%%%%%%%%%%%%%%%%%%%%%%%%%%
\begin{frame}{Relation between diagonal circulant matrices and low rank matrices}
%%%%%%%%%%%%%%%%%%%%%%%%%%%%%%%%%%%%%%%%%%%%%%%%%%%%%%%%%%%%%%%%%%%%%%%%%%%%%%%

  \begin{minipage}{\textwidth}
    \textbf{Question:} Can we devise a expressivity result with a product of less than $n + 1$ diagonal-circulant factors ?
  \end{minipage}

  \visible<2->{
    \begin{theorem}[Rank-based circulant decomposition]
      For every matrix $\Amat \in \Cbb^{n \times n}$ of rank $r$, for any $\epsilon > 0$, there exists a sequence of $2r+1$ diagonal-circulant factors such that: 
      \begin{equation}
	\norm{\prod_{i=1}^{2r+1} \Dmat^{(i)} \Cmat^{(i)} - \Amat}_{\mathrm{F}} < \epsilon \enspace.
      \end{equation}
    \end{theorem}
  }

  \visible<3->{
    \textbf{Remark}: If the number of diagonal-circulant factors is set to a value $k$, we can represent all linear transform whose rank is $\frac{k - 1}{2}$.
  }

\end{frame}


%%%%%%%%%%%%%%%%%%%%%%%%%%%%%%%%%%%%%%%%%%%%%%%%%%%%%%%%%%%%%%%%%%%%%%%%%%%%%%%
\begin{frame}{Idea of the proof}
%%%%%%%%%%%%%%%%%%%%%%%%%%%%%%%%%%%%%%%%%%%%%%%%%%%%%%%%%%%%%%%%%%%%%%%%%%%%%%%

  Let $\Amat \in \Cbb^{n \times n}$ be a matrix of rank $r$ and let $\Amat = \Umat \boldsymbol{\Sigma} \Vmat^*$ be the singular value decomposition of the matrix $\Amat$
  \begin{equation*}
    \Amat = \scalebox{0.5}{
\begin{tikzpicture}[
  baseline,
  every left delimiter/.style={xshift=+0.85em},
  every right delimiter/.style={xshift=-0.5em},
  mymat/.style={
    matrix of math nodes,
    ampersand replacement=\&,
    left delimiter=(,
    right delimiter=),
    nodes in empty cells,
    nodes={
     outer sep=-0.3mm,
     text depth=0.5ex,
     text height=0.8ex,
     text width=0.8em,
     align=center}
  }
  ]
  \matrix[mymat] (matu) {
    \& \& \& \& \& \\
    \& \& \& \& \& \\
    \& \& \& \& \& \\
    \& \& \& \& \& \\
    \& \& \& \& \& \\
    \& \& \& \& \& \\
  };
  \node at ([shift={(10pt,-5pt)}]matu-3-2.west) {$\cdots$};
  \node at ([shift={(10pt,-5pt)}]matu-3-5.west) {$\cdots$};
  \foreach \Columna/\Valor in {1/1,3/r,4/{r+1},6/n} {
    \def\mytestcolor{none}
    \ifnum\Columna=1\relax\def\mytestcolor{myred}\fi
    \ifnum\Columna=3\relax\def\mytestcolor{RoyalBlue}\fi
    \draw [fill=\mytestcolor] (matu-1-\Columna.north) rectangle ([xshift=4pt]matu-6-\Columna.south);
    % \node[above] at ([xshift=2pt]matu-1-\Columna.north west) {$\uvec_{\Valor}$};
  }
  % \draw[decorate,decoration={brace,mirror,raise=3pt}] 
  %   (matu-6-1.south west) -- 
  %    node[below=4pt] {$\Mcol(\Amat)$}
  %   ([xshift=4pt]matu-6-3.south west);
  % \draw[decorate,decoration={brace,mirror,raise=3pt}] 
  %   (matu-6-4.south west) -- 
  %    node[below=4pt] {$\Mnull(\Amat)$}
  %   ([xshift=4pt]matu-6-6.south west);
  \draw[decorate,decoration={brace,mirror,raise=5pt}]
     (matu-1-3.north east) -- 
      node[above=6pt] {$r$ columns}
     (matu-1-1.north west);
  \matrix[mymat,right=10pt of matu] (matsigma) {
    \sigma_{1} \& \& \& \& \& \\
    \& \ddots \& \& \& \& \\
    \& \& \sigma_{r} \& \& \& \\
    \& \& \& 0 \& \& \\
    \& \& \& \& \ddots \& \\
    \& \& \& \& \& 0 \\
  };
  \matrix[mymat,right=10pt of matsigma] (matv) {
    \& \& \& \& \& \\
    \& \& \& \& \& \\
    \& \& \& \& \& \\
    \& \& \& \& \& \\
    \& \& \& \& \& \\
    \& \& \& \& \& \\
  };
  \node at ([shift={(2pt,-5pt)}]matv-2-3.east) {$\cdots$};
  \node at ([shift={(2pt,-5pt)}]matv-5-3.east) {$\cdots$};
  \foreach \Fila/\Valor in {1/1,3/r,4/{r+1},6/n} {
    \def\mytestcolor{none}
    \ifnum\Fila=1\relax\def\mytestcolor{OrangePSL}\fi
    \ifnum\Fila=3\relax\def\mytestcolor{color2}\fi
    \draw[fill=\mytestcolor] ([yshift=-6pt]matv-\Fila-1.north west) rectangle ([yshift=-10pt]matv-\Fila-6.north east);
  }

  \matrix[mymat,below=15pt of matu] (mat1) {
    \& \& \& \& \& \\
    \& \& \& \& \& \\
    \& \& \& \& \& \\
    \& \& \& \& \& \\
    \& \& \& \& \& \\
    \& \& \& \& \& \\
  };
  \matrix[mymat,right=10pt of mat1] (mat2) {
    {\color{color1}{\absf}} \& {\color{color2}{\bbsf}} \& {\color{color3}{\cbsf}} \& {\color{color4}{\dbsf}} \& {\color{color5}{\ebsf}} \& {\color{color6}{\fbsf}} \\
    {\color{color6}{\fbsf}} \& {\color{color1}{\absf}} \& {\color{color2}{\bbsf}} \& {\color{color3}{\cbsf}} \& {\color{color4}{\dbsf}} \& {\color{color5}{\ebsf}} \\
    {\color{color5}{\ebsf}} \& {\color{color6}{\fbsf}} \& {\color{color1}{\absf}} \& {\color{color2}{\bbsf}} \& {\color{color3}{\cbsf}} \& {\color{color4}{\dbsf}} \\
    {\color{color4}{\dbsf}} \& {\color{color5}{\ebsf}} \& {\color{color6}{\fbsf}} \& {\color{color1}{\absf}} \& {\color{color2}{\bbsf}} \& {\color{color3}{\cbsf}} \\
    {\color{color3}{\cbsf}} \& {\color{color4}{\dbsf}} \& {\color{color5}{\ebsf}} \& {\color{color6}{\fbsf}} \& {\color{color1}{\absf}} \& {\color{color2}{\bbsf}} \\
    {\color{color2}{\bbsf}} \& {\color{color3}{\cbsf}} \& {\color{color4}{\dbsf}} \& {\color{color5}{\ebsf}} \& {\color{color6}{\fbsf}} \& {\color{color1}{\absf}} \\
  };


  \draw [color6,fill=color6, rounded corners=1mm, opacity=0.5] (mat2-1-4.north west) rectangle (mat2-1-4.south east);
  \draw [color6,fill=color6, rounded corners=1mm, opacity=0.5] (mat2-3-2.north west) rectangle (mat2-3-2.south east);
  \draw [color6,fill=color6, rounded corners=1mm, opacity=0.5] (mat2-2-1.north west) rectangle (mat2-2-1.south east);
  \draw [color6,fill=color6, rounded corners=1mm, opacity=0.5] (mat2-4-3.north west) rectangle (mat2-4-3.south east);

  \draw [color5,fill=color5, rounded corners=1mm, opacity=0.5] (mat2-1-4.north west) rectangle (mat2-1-4.south east);
  \draw [color5,fill=color5, rounded corners=1mm, opacity=0.5] (mat2-3-2.north west) rectangle (mat2-3-2.south east);
  \draw [color5,fill=color5, rounded corners=1mm, opacity=0.5] (mat2-2-1.north west) rectangle (mat2-2-1.south east);
  \draw [color5,fill=color5, rounded corners=1mm, opacity=0.5] (mat2-4-3.north west) rectangle (mat2-4-3.south east);

  \draw [color4,fill=color4, rounded corners=1mm, opacity=0.5] (mat2-1-4.north west) rectangle (mat2-1-4.south east);
  \draw [color4,fill=color4, rounded corners=1mm, opacity=0.5] (mat2-3-2.north west) rectangle (mat2-3-2.south east);
  \draw [color4,fill=color4, rounded corners=1mm, opacity=0.5] (mat2-2-1.north west) rectangle (mat2-2-1.south east);
  \draw [color4,fill=color4, rounded corners=1mm, opacity=0.5] (mat2-4-3.north west) rectangle (mat2-4-3.south east);

  \draw [color3,fill=color3, rounded corners=1mm, opacity=0.5] (mat2-1-3.north west) rectangle (mat2-1-3.south east);
  \draw [color3,fill=color3, rounded corners=1mm, opacity=0.5] (mat2-2-4.north west) rectangle (mat2-2-4.south east);
  \draw [color3,fill=color3, rounded corners=1mm, opacity=0.5] (mat2-3-1.north west) rectangle (mat2-3-1.south east);
  \draw [color3,fill=color3, rounded corners=1mm, opacity=0.5] (mat2-4-2.north west) rectangle (mat2-4-2.south east);

  \draw [color2,fill=color2, rounded corners=1mm, opacity=0.5] (mat2-1-2.north west) rectangle (mat2-1-2.south east);
  \draw [color2,fill=color2, rounded corners=1mm, opacity=0.5] (mat2-2-3.north west) rectangle (mat2-2-3.south east);
  \draw [color2,fill=color2, rounded corners=1mm, opacity=0.5] (mat2-3-4.north west) rectangle (mat2-3-4.south east);
  \draw [color2,fill=color2, rounded corners=1mm, opacity=0.5] (mat2-4-1.north west) rectangle (mat2-4-1.south east);

  \draw [color1,fill=color1, rounded corners=1mm, opacity=0.5] (mat2-1-1.north west) rectangle (mat2-1-1.south east);
  \draw [color1,fill=color1, rounded corners=1mm, opacity=0.5] (mat2-2-2.north west) rectangle (mat2-2-2.south east);
  \draw [color1,fill=color1, rounded corners=1mm, opacity=0.5] (mat2-3-3.north west) rectangle (mat2-3-3.south east);
  \draw [color1,fill=color1, rounded corners=1mm, opacity=0.5] (mat2-4-4.north west) rectangle (mat2-4-4.south east);
  \draw [color1,fill=color1, rounded corners=1mm, opacity=0.5] (mat2-5-5.north west) rectangle (mat2-5-5.south east);
  \draw [color1,fill=color1, rounded corners=1mm, opacity=0.5] (mat2-6-6.north west) rectangle (mat2-6-6.south east);


  \matrix[mymat,right=10pt of mat2] (mat3) {
    1 \& \& \& \& \& \\
    \& \ddots \& \& \& \& \\
    \& \& 1 \& \& \& \\
    \& \& \& 0 \& \& \\
    \& \& \& \& \ddots \& \\
    \& \& \& \& \& 0 \\
  };
  \draw[decorate,decoration={brace,mirror,raise=10pt,aspect=0.85}]
     (mat3-1-6.north east) -- 
      node[below=3pt] { }
     (mat1-1-1.north west);
  % \draw[decorate,decoration={brace,raise=0pt}] 
  %   ($(mat3-1-1.north)+(.0,.0)$) -- ($(mat3-3-3.north east)+(0.1,-0.17)$)
  %   node[above=5pt,midway,sloped] {$r$};

  


\end{tikzpicture}
}
  \end{equation*}


\end{frame}



%%%%%%%%%%%%%%%%%%%%%%%%%%%%%%%%%%%%%%%%%%%%%%%%%%%%%%%%%%%%%%%%%%%%%%%%%%%%%%%
\begin{frame}{Diagonal-Circulant Layer}
%%%%%%%%%%%%%%%%%%%%%%%%%%%%%%%%%%%%%%%%%%%%%%%%%%%%%%%%%%%%%%%%%%%%%%%%%%%%%%%

  We replace the weight matrices of Fully-Connected layers by a product of Diagonal and Circulant matrices :

  \begin{equation}
    \xvec \mapsto \left[ \orange{ \prod_{i=1}^{k} \Dmat^{(i)} \Cmat^{(i)} } \right] \xvec + \bvec
  \end{equation}
  where
  \begin{itemize}
    \item[$\bullet$] $\xvec \in \Rbb^n$, $\bvec \in \Rbb^n$,
    \item[$\bullet$] $\Dmat^{(i)} \in \Rbb^{n \times n}$ is a diagonal matrix,
    \item[$\bullet$] $\Cmat^{(i)} \in \Rbb^{n \times n}$ is a circulant matrix,
    \item[$\bullet$] $k$ is a user defined parameter controlling the expressivity.
  \end{itemize}

  \pause
  \textbf{Remark:} Instead of defining a parameter $k$ for each Diagonal-Circulant layer, we set them all to $k = 1$ and we adjust the depth of the network.

\end{frame}



%%%%%%%%%%%%%%%%%%%%%%%%%%%%%%%%%%%%%%%%%%%%%%%%%%%%%%%%%%%%%%%%%%%%%%%%%%%%%%%
\begin{frame}{Expressive Power of Diagonal-Circulant Neural Network}
%%%%%%%%%%%%%%%%%%%%%%%%%%%%%%%%%%%%%%%%%%%%%%%%%%%%%%%%%%%%%%%%%%%%%%%%%%%%%%%

  \begin{theorem}[Rank-based expressive power of DCNNs]
    Let $N$ be a neural network of width $n$, depth $p$ and a sum of ranks of the weight matrices $k$.
    Then, for any $\epsilon>0$, there exists a DCNN $N'$ of width $n$ such that 
    \begin{equation}
      \norm{N(\xvec) - N'(\xvec) }_2 < \epsilon
    \end{equation}
    and the depth of $N'$ is bounded by $9k$.
  \end{theorem}

  \todo{xxx}

  Let $\Rcal_{k,n}$ be the set of all functions $f:\mathbb{R}^{n}\rightarrow\mathbb{R}^{n}$ representable by deep neural networks of total rank at most $k$ and let $\Ccal_{l,n}$ the set of all functions $f:\Rbb^{n} \rightarrow \Rbb^{n}$ representable by deep diagonal-circulant networks of depth at most $l$, then:
    \begin{align*}
      \forall k,\exists l,\forall n\, & \mathcal{R}_{k,n}\varsubsetneq\mathcal{C}_{l,n} \\
      \forall l,\nexists k,\forall n\, & \mathcal{C}_{l,n}\subseteq\mathcal{R}_{k,n}
    \end{align*}

  As we can see, the set $\Rcal_{k,n}$ of all the functions representable by a deep neural network of total rank $k$ is strictly included in the set $\Ccal_{9k}$ of all diagonal-circulant neural networks of depth $9k$. 

  \begin{figure}[htb]
    \scalebox{0.65}{\tikzset{%
  >={Latex[width=2mm,length=2mm]},
            base/.style = {rectangle, draw=black, text centered, font=\sffamily},
           other/.style = {base, fill=none,  minimum width=1.7cm, minimum height=0.7cm},
         ellipse/.style = {base}
}
\begin{tikzpicture}[every node/.style={fill=white, font=\sffamily}, align=center,scale=0.6]

    \draw (0,0) circle (2.5cm);
    \draw (0,0) circle (2.0cm);
    \draw (0,0) circle (1.5cm);
    \draw (0,0) circle (1.0cm);
    \draw (0,0) circle (0.5cm);
    
    \draw[ellipse, rotate=30, fill=gray, opacity=0.5] (0.1, -1.1) ellipse (2.0cm and 0.9cm);
    \draw[ellipse, rotate=30, fill=gray, opacity=0.5] (0.0, -0.8) ellipse (1.0cm and 0.45cm);

    \node[other, draw=none] at (0.20, 0.20) {$\mathcal{C}_{1,n}$};
    \node[other, draw=none] at (0.55, 0.55) {$\iddots$};
    \node[other, draw=none] at (0.90, 0.90) {$\mathcal{C}_{9,n}$};
    \node[other, draw=none] at (1.25, 1.25) {$\iddots$};
    \node[other, draw=none] at (1.60, 1.60) {$\mathcal{C}_{18,n}$};
    
    \node[other, draw=none] at (0.4, -0.6) {$\mathcal{R}_{1,n}$};
    \node[other, draw=none] at (1.0, -1.4) {$\mathcal{R}_{2,n}$};

\end{tikzpicture}}
  \end{figure}

  
\end{frame}


%%%%%%%%%%%%%%%%%%%%%%%%%%%%%%%%%%%%%%%%%%%%%%%%%%%%%%%%%%%%%%%%%%%%%%%%%%%%%%%
\begin{frame}{Training of Diagonal-Circulant Neural Networks}
%%%%%%%%%%%%%%%%%%%%%%%%%%%%%%%%%%%%%%%%%%%%%%%%%%%%%%%%%%%%%%%%%%%%%%%%%%%%%%%

  Training diagonal circulant neural networks is hard !!

  other approaches

\end{frame}



%%%%%%%%%%%%%%%%%%%%%%%%%%%%%%%%%%%%%%%%%%%%%%%%%%%%%%%%%%%%%%%%%%%%%%%%%%%%%%%
\begin{frame}{Training of Diagonal-Circulant Neural Networks}
%%%%%%%%%%%%%%%%%%%%%%%%%%%%%%%%%%%%%%%%%%%%%%%%%%%%%%%%%%%%%%%%%%%%%%%%%%%%%%%

  \begin{block}{Large Scale Video Classification with the \yt dataset}
    \begin{itemize}
      \item 8 millions embedded audio \& video frames
      \item 3200 classes
    \end{itemize}
  \end{block}

  % The network randomly samples video and audio frames from the input. The sample goes through an embedding layer and is reduced with a Fully Connected layer. The results are then concatenated and classified with a Mixture-of-Experts and a Context Gating layer.
  State-of-the-art architecture for video classification ({\color{SkyBlue}{\cite{miech2017learnable}}}).
  \begin{figure}[htb]
    \scalebox{0.65}{\tikzset{%
  >={Latex[width=2mm,length=2mm]},
  % Specifications for style of nodes:
            base/.style = {rectangle, draw=black, text centered, font=\sffamily},
             box/.style = {base, rounded corners, text depth=3cm, minimum height=4cm, minimum width=3cm},
     transparent/.style = {rectangle, draw=black},
       circulant/.style = {base, fill=yellow!30},
       embedding/.style = {base, fill=blue!30, minimum width=2.5cm, minimum height=1cm},
           other/.style = {base, fill=white!30,  minimum width=2cm, minimum height=1cm},
              fc/.style = {base, fill=orange!30, minimum width=1.5cm, minimum height=1cm},
          gating/.style = {base, fill=green!30, minimum width=2cm, text width=2cm, minimum height=1cm},
             moe/.style = {base, fill=purple!30, minimum width=1.5cm, minimum height=1cm},
}

\begin{tikzpicture}[every node/.style={fill=white, font=\sffamily}, align=center]

  % \draw (0.0, +2.)  node [other, draw=none, opacity=0, text opacity=1] {\textbf{Embedding}};
  % \draw (+3.7, +2.)  node [other, draw=none, opacity=0, text opacity=1] {\textbf{Dim Reduction}};
  % \draw (+8.0, +2.)  node [other, draw=none, opacity=0, text opacity=1] {\textbf{Classification}};
  \draw (0.0, +2.)  node [other, draw=none, opacity=0, text opacity=1] {\textbf{Layer 1}};
  \draw (+3.7, +2.)  node [other, draw=none, opacity=0, text opacity=1] {\textbf{Layer 2}};
  \draw (+8.0, +2.)  node [other, draw=none, opacity=0, text opacity=1] {\textbf{Layer 3}};

  \draw (0, +0.8)  node [embedding] {Video};
  \draw (0, -0.8)  node [embedding] {Audio};

  \draw (+2.5, +0.8)  node (fc) [fc] {FC};
  \draw (+2.5, -0.8)  node (fc) [fc] {FC};

  \draw (+4.75, 0)  node (fc) [other] {concat};
  \draw (+7.0, 0)  node (moe) [moe] {MoE};
  \draw (+9.25, 0)  node (gating2) [gating] {Context Gating};
 
  \draw (+1.5, +2) [dashed] -- (+1.5, -1.7);
  \draw (+6, +2) [dashed] -- (+6, -1.7);
  
  % \draw (3.5, -2.6)  node [other, draw=none, opacity=0, text opacity=1] {\textbf{use of Diagonal-Circulant layers}};
  % \draw (0.0, -1.5) -- (2.7, -2.3);
  % \draw (2.5, -1.5) -- (3.3, -2.3);
  % \draw (7.0, -0.8) -- (4.0, -2.3);
  
\end{tikzpicture}
}
  \end{figure}
  $\Rightarrow$ This architecture has 5.7 millions parameters.

\end{frame}


%%%%%%%%%%%%%%%%%%%%%%%%%%%%%%%%%%%%%%%%%%%%%%%%%%%%%%%%%%%%%%%%%%%%%%%%%%%%%%%
\begin{frame}{Effect of Diagonal-Circulant layers}
%%%%%%%%%%%%%%%%%%%%%%%%%%%%%%%%%%%%%%%%%%%%%%%%%%%%%%%%%%%%%%%%%%%%%%%%%%%%%%%

  \begin{minipage}{\textwidth}
    \centering
    Graph representing the trade-off between accuracy and compression rate.
  \end{minipage}
  \vspace{0.2cm}

  \only<1>{
    \begin{minipage}{\textwidth}
      \centering
      \scalebox{0.8}{\begin{tikzpicture}
\begin{axis}[
    width=0.85\textwidth,
    height=0.6\textwidth,
    xlabel={Epochs},
    ylabel={Accuracy GAP},
    xmin=0, xmax=7,
    ymin=0.63, ymax=0.87,
    xtick={0,1,2,3,4,5,6,7},
    ytick={0.63, 0.66, 0.69, 0.72, 0.75, 0.78, 0.81, 0.84, 0.87},
    ymajorgrids=true,
    grid style=dashed,
	]
  \addplot[color=myred!100, thick, dashed] table [y=gap, x=epoch]{data/layers/dense.dat};
  % \addplot[color=blue!30,   thick] table [y=gap, x=epoch]{data/layers/compact_dbof.dat};
  % \addplot[color=yellow!30, thick] table [y=gap, x=epoch]{data/layers/compact_fc.dat};
  % \addplot[color=green!100, thick] table [y=gap, x=epoch]{data/layers/compact_moe.dat};

  % \draw [color=myred!100, thick, <-] (axis cs:3.0,0.843) -- +(+10pt,+10pt) node[right] {Original};
  % \draw [color=black!30,  thick, <-] (axis cs:2.0,0.834) -- +(-10pt,+10pt) node[left]  {9.2\%};
  % \draw [color=black!30,  thick, <-] (axis cs:5.2,0.830) -- +(+10pt,-10pt) node[right] {18.4\%};
  % \draw [color=black!100, thick, <-] (axis cs:5.0,0.800) -- +(+10pt,-10pt) node[right] {72.0\%};


\end{axis}
\end{tikzpicture}
}
    \end{minipage}

    \vspace{0.3cm}
    \begin{minipage}{\textwidth}
      \centering
      The original architecture achieve an accuracy GAP of 84\%.
    \end{minipage}
  }

  \only<2>{
    \begin{minipage}{\textwidth}
      \centering
      \scalebox{0.8}{\begin{tikzpicture}
\begin{axis}[
    width=0.85\textwidth,
    height=0.6\textwidth,
    xlabel={Epochs},
    ylabel={GAP},
    xmin=0, xmax=7,
    ymin=0.63, ymax=0.87,
    xtick={0,1,2,3,4,5,6,7},
    ytick={0.63, 0.66, 0.69, 0.72, 0.75, 0.78, 0.81, 0.84, 0.87},
    ymajorgrids=true,
    grid style=dashed,
	]
  \addplot[color=myred!100, thick, dashed] table [y=gap, x=epoch]{data/layers/dense.dat};
  \addplot[color=NavyBlue!100, thick] table [y=gap, x=epoch]{data/layers/compact_fc.dat};
  % \addplot[color=yellow!100,  thick] table [y=gap, x=epoch]{data/layers/compact_dbof.dat};
  % \addplot[color=green!100, thick] table [y=gap, x=epoch]{data/layers/compact_moe.dat};

  % \draw [color=myred!100, thick, <-] (axis cs:2.8,0.835) -- +(+10pt,-10pt) node[right] {Original};
  % \draw [color=myred!100, thick, <-] (axis cs:3.0,0.843) -- +(+10pt,+10pt) node[right] {Original};
  \draw [color=NavyBlue!100,  thick, <-] (axis cs:2.0,0.834) -- +(-10pt,+10pt) node[left]  {9.2\%};
  % \draw [color=black!30,  thick, <-] (axis cs:5.2,0.830) -- +(+10pt,-10pt) node[right] {18.4\%};
  % \draw [color=black!100, thick, <-] (axis cs:5.0,0.800) -- +(+10pt,-10pt) node[right] {72.0\%};


\end{axis}
\end{tikzpicture}
}
    \end{minipage}

    \vspace{0.3cm}
    \begin{minipage}{\textwidth}
      \centering
      We achieve \orangebold{9.2\% compression rate} with \orangebold{no loss} in accuracy.
    \end{minipage}
  }

  \only<3>{
    \begin{minipage}{\textwidth}
      \centering
      \scalebox{0.8}{\begin{tikzpicture}
\begin{axis}[
    width=0.85\textwidth,
    height=0.6\textwidth,
    xlabel={Epochs},
    ylabel={Accuracy GAP},
    xmin=0, xmax=7,
    ymin=0.63, ymax=0.87,
    xtick={0,1,2,3,4,5,6,7},
    ytick={0.63, 0.66, 0.69, 0.72, 0.75, 0.78, 0.81, 0.84, 0.87},
    ymajorgrids=true,
    grid style=dashed,
	]
  \addplot[color=myred!100,  thick, dashed] table [y=gap, x=epoch]{data/layers/dense.dat};
  \addplot[color=NavyBlue!100, thick] table [y=gap, x=epoch]{data/layers/compact_fc.dat};
  \addplot[color=OrangePSL!100, thick] table [y=gap, x=epoch]{data/layers/compact_dbof.dat};
  % \addplot[color=green!100, thick] table [y=gap, x=epoch]{data/layers/compact_moe.dat};

  % \draw [color=myred!100, thick, <-] (axis cs:2.8,0.835) -- +(+10pt,-10pt) node[right] {Original};
  % \draw [color=myred!100, thick, <-] (axis cs:3.0,0.843) -- +(+10pt,+10pt) node[right] {Original};
  % \draw [color=NavyBlue!30,  thick, <-] (axis cs:2.0,0.834) -- +(-10pt,+10pt) node[left]  {9.2\%};
  \draw [color=OrangePSL!100, thick, <-] (axis cs:5.2,0.830) -- +(+10pt,-10pt) node[right] {18.4\%};
  % \draw [color=black!100, thick, <-] (axis cs:5.0,0.800) -- +(+10pt,-10pt) node[right] {72.0\%};



\end{axis}
\end{tikzpicture}
}
    \end{minipage}

    \vspace{0.3cm}
    \begin{minipage}{\textwidth}
      \centering
      We achieve \orangebold{18\% compression rate} with a loss of \orangebold{2 points} in accuracy.
    \end{minipage}
  }

  \only<4>{
    \begin{minipage}{\textwidth}
      \centering
      \scalebox{0.8}{\begin{tikzpicture}
\begin{axis}[
    width=0.85\textwidth,
    height=0.6\textwidth,
    xlabel={Epochs},
    ylabel={Accuracy GAP},
    xmin=0, xmax=7,
    ymin=0.63, ymax=0.87,
    xtick={0,1,2,3,4,5,6,7},
    ytick={0.63, 0.66, 0.69, 0.72, 0.75, 0.78, 0.81, 0.84, 0.87},
    ymajorgrids=true,
    grid style=dashed,
	]
  \addplot[color=myred!100,      thick, dashed] table [y=gap, x=epoch]{data/layers/dense.dat};
  \addplot[color=NavyBlue!100,   thick] table [y=gap, x=epoch]{data/layers/compact_fc.dat};
  \addplot[color=OrangePSL!100,  thick] table [y=gap, x=epoch]{data/layers/compact_dbof.dat};
  \addplot[color=OliveGreen!100, thick] table [y=gap, x=epoch]{data/layers/compact_moe.dat};

  % \draw [color=myred!100,      thick, <-] (axis cs:2.8,0.835) -- +(+10pt,-10pt) node[right] {Original};
  % \draw [color=myred!100, thick, <-] (axis cs:3.0,0.843) -- +(+10pt,+10pt) node[right] {Original};
  % \draw [color=NavyBlue!30,    thick, <-] (axis cs:2.0,0.834) -- +(-10pt,+10pt) node[left]  {9.2\%};
  % \draw [color=OrangePSL!30,   thick, <-] (axis cs:5.2,0.830) -- +(+10pt,-10pt) node[right] {18.4\%};
  \draw [color=OliveGreen!100, thick, <-] (axis cs:5.0,0.800) -- +(+10pt,-10pt) node[right] {72.0\%};



\end{axis}
\end{tikzpicture}
}
    \end{minipage}

    \vspace{0.3cm}
    \begin{minipage}{\textwidth}
      \centering
      We achieve \orangebold{72\% compression rate} with a loss of only \orangebold{4 points} in accuracy.
    \end{minipage}
  }


\end{frame}


